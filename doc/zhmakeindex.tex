\documentclass[UTF8,hyperref,nofonts]{ctexart}
\xeCJKsetup{AutoFakeBold=false,PunctStyle=hangmobanjiao}
\setCJKmainfont[ItalicFont=FandolKai]{FandolSong}
\setCJKsansfont{FandolHei}
\setCJKmonofont{FandolFang}

\usepackage[a4paper,centering,margin=1.5in]{geometry}

\CTEXsetup[format=\bfseries\Large]{section}

\bibliographystyle{plain}

\usepackage{makeidx}
\makeindex

\usepackage{shortvrb}
\MakeShortVerb\"

\usepackage{tabularx}

\usepackage{xspace}

\newcommand\pkg[1]{\textsf{#1}\index{#1@\textsf{#1}}}
\newcommand\zhm{\pkg{zhmakeindex}\xspace}
\newcommand\meta[1]{$\langle\textnormal{\itshape#1}\rangle$}
\newcommand\optional[1]{\textnormal{[}#1\textnormal{]}}
\def\defchar#1{\begingroup\lccode`~=`#1\lowercase{\endgroup\def~}}
\newenvironment{syntax}{%
  \quote\ttfamily
  \catcode`<\active  \defchar<##1>{\meta{##1}}
  \catcode`[\active  \defchar[##1]{\optional{##1}}
}{\endquote}
\newcommand\email{\nolinkurl}

\title{\zhm\thanks{版本 1.0} 中文索引处理程序}
\author{刘海洋}
\date{2014 年 2 月 6 日}

\begin{document}

\maketitle

\section{命令行}

\begin{syntax}
\halign{#&#\hfil\cr
zhmakeindex &[-c] [-i] [-o~<ind>] [-q] [-r] [-s~<sty>] [-t~<log>]\cr
            &[-enc~<enc>] [-senc~<senc>] [-strict] [-z~<sort>]\cr
            &[<idx0> <idx1> <idx2> ...]\cr
}
\end{syntax}

\section{简介}

\zhm 是一个通用的中文多级索引处理程序,它从一个或多个输入文件读入索引项,将其
内容按指定的方式分组、排序,然后按格式将整理好的索引输出到文件。索引项可以有 3
个级别(0, 1, 2)的嵌套。\zhm 主要用于 \LaTeX{} 索引的处理,其功能和用法与
\pkg{makeindex}\cite{Rodgers1991} 相似,并支持中文的分组与排序。

输入与输出文件的格式由一个格式文件确定。默认的输入/输出是 ".idx"/".ind" 格式
的,即 \LaTeX{} 格式的索引文件。

如果没有显式指定,第一个输入文件(\meta{idx0})的主文件名将用于确定输出和日志
文件的主文件名。对每个输入文件名 \meta{idx0}, \meta{idx1}, \ldots, \zhm 会首先
查找这个名字的文件;如果找不到且文件名没有后缀,则加上 ".idx" 后缀查找此文件;
如果还找不到文件,\zhm 则会中止。

如果只有一个输入文件,并且没有显式用 "-s" 选项指定格式文件,\zhm 会使用后缀为
".mst" 的默认格式文件(如果存在的话)。


\section{选项说明}

\subsection{与 \pkg{makeindex} 相同的选项}

\begin{description}
  \item[-c] 压缩索引项排序键前后的空格。默认情况下,索引键的空格会被保留。
  \item[-i] 从标准输入流("stdin")读入索引项。如果使用了该选项,并且没有使用
    "-o" 选项,则排序后的索引将输出到标准输出流("stdout")。
  \item[-o~\meta{ind}] 设置输出索引文件为 \meta{ind}。如果没有指定该选项,默认
    的输出文件名是第一个输入文件 \meta{idx0} 的主文件名加上 ".ind" 的扩展名。
  \item[-q] 静默模式,不向标准错误流("stderr")显示信息。默认情况下处理过程与
    错误信息会同时在 "stderr" 与日志文件中输出。
  \item[-r] 禁止隐式页码区间构造,要求页码区间必须使用显式区间符号生成。见
    第~\ref{sec:sepcialeffects} 节的说明。默认情况下,三个或三个以上连续的页码
    会自动合并为一个页码区间(如 1--5)。
  \item[-s~\meta{sty}] 设置 \meta{sty} 为格式文件。没有默认值。\zhm 会首先在当
    前目录查找格式文件,如果找不到则调用 \TeX{} 发行版的 "kpathsea" 库在 TEXMF
    树中查找。
  \item[-t~\meta{log}] 设置 \meta{log} 为日志文件。默认情况下,会使用第一个输
    入文件 \meta{idx0} 的主文件名加上 ".ilg" 后缀作为日志文件。
\end{description}

\subsection{\zhm 独有的选项}

\begin{description}
  \item[-enc~\meta{enc}] 设置输入输出文件的编码为 \meta{enc}。可选的编码包括
    "UTF-8", "UTF-16", "GB18030", "GBK" 和 "Big5",不区分大小写。默认使用
    UTF-8 编码。
  \item[-senc~\meta{senc}] 设置读入格式文件的编码为 \meta{senc}。可选的编码与
    "-enc" 选项相同。默认使用 UTF-8 编码。
  \item[-strict] 严格区分不同 encapsulated 命令的页码。默认情况下会将
  \item[-z~\meta{sort}] 设置中文排序分组方式为 \meta{sort}。可选的中文排序分组
    方式如表~\ref{tab:sort} 所示。默认按拼音排序分组。
\begin{table}[htbp]
\centering
\begin{tabularx}{\linewidth}{llX}
\hline
选项 & 别名 & 意义 \\
\hline
"pinyin" & "reading" & 按第一个汉字拼音首字母与西文混合分组,汉字按拼音
  排序。 \\
"bihua" & "stroke" & 汉字按笔画数与西文各自分组,按笔画数和笔顺排序。 \\
"bushou" & "radical" & 汉字按康熙字典部首与西文各自分组,按部首和除部首笔画数
  排序。 \\
\hline
\end{tabularx}
\caption{\zhm 支持的中文排序分组方式}\label{tab:sort}
\end{table}
\end{description}

\subsection{未实现的选项}

\zhm 没有实现 \pkg{makeindex} 的 "-g", "-l", "-p", "-L", "-T" 选项。其中,选项
"-p" 依赖对 \TeX{} 编译的日志文件的解析,可能在未来版本实现;另外几个语言相关
的排序选项("-g" 德文,"-T" 泰文,"-L" 做系统 locale 选择,"-l" 有关西文单词排
序)对中文索引意义较小,不在 \zhm 中实现。

\section{格式文件}

\zhm 的格式文件与标准 \pkg{makeindex} 基本上完全兼容,同时增加了少量控制中文
分组输出的格式。可以在 \zhm 中使用标准 \pkg{makeindex} 的格式文件。

\subsection{与 \pkg{makeindex} 兼容的格式}

\subsection{\zhm 特有的格式}

\subsection{未实现的 \pkg{makeindex} 格式}

\section{排序细节}

\section{特殊效果}
\label{sec:sepcialeffects}

\section{与 \pkg{makeindex} 的比较}

\section{版权与许可}
\index{版权}\index{许可}

版权所有:2014 年,刘海洋 \email{leoliu.pku@gmail.com}

\index{LPPL}
本作品可在《the \LaTeX{} Project Public License》1.3 或更高版本的条件下发布与
修改。最新版本的 LPPL 许可证可以在
\begin{quote}
  \url{http://www.latex-project.org/lppl.txt}
\end{quote}
下载;该许可证同时也包含在所有最新的 \LaTeX{} 发行版中。

本作品目前处于 LPPL 维护状态“author-maintained”。

当前维护者是刘海洋。

\index{源文件}
本作品包括 \zhm 的程序及文档,由如下源文件:
\begin{verbatim}
dist.cmd
examples
input.go
main.go
numberedreader.go
output.go
radicalstrokes.go
radical_collator.go
readings.go
reading_collator.go
sorter.go
strokes.go
stroke_collator.go
style.go
style_test.go
doc/zhmakeindex.tex
kpathsea/dynamic_other.go
kpathsea/dynamic_windows_386.go
kpathsea/kpathsea.go
maketables/make-table.cmd
maketables/maketables.go
\end{verbatim}
以及编译源文件得到的二进制文件 \path{zhmakeindex.exe}、PDF 文档
\path{zhmakeindex.pdf} 组成。

\index{Unicode}
大部分汉字数据来自 Unicode 项目(\url{http://www.unicode.org/}):
\begin{verbatim}
maketables/CJKRadicals.txt
maketables/Unihan_DictionaryLikeData.txt
maketables/Unihan_RadicalStrokeCounts.txt
maketables/Unihan_Readings.txt
\end{verbatim}

\index{海峰五笔}
部分字形数据来自海峰五笔项目(\url{http://okuc.net/sunwb/}):
\begin{verbatim}
maketables/sunwb_strokeorder.txt
\end{verbatim}

\bibliography{zhmakeindex}

\printindex

\end{document}

vim:tw=78
