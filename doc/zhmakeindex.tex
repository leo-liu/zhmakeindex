\documentclass[UTF8,hyperref]{ctexart}

\usepackage[a4paper,centering,margin=1.5in]{geometry}

\CTEXsetup[format=\bfseries\Large]{section}

\bibliographystyle{plain}

\usepackage{makeidx}
\makeindex

\usepackage{shortvrb}
\MakeShortVerb\"

\usepackage{xspace}

\newcommand\pkg[1]{\textsf{#1}\index{#1@\textsf{#1}}}
\newcommand\zhm{\pkg{zhmakeindex}\xspace}
\newcommand\meta[1]{$\langle\textnormal{\itshape#1}\rangle$}
\newcommand\optional[1]{\textnormal{[}#1\textnormal{]}}
\def\defchar#1{\begingroup\lccode`~=`#1\lowercase{\endgroup\def~}}
\newenvironment{syntax}{%
  \quote\ttfamily
  \catcode`<\active  \defchar<##1>{\meta{##1}}
  \catcode`[\active  \defchar[##1]{\optional{##1}}
}{\endquote}
\newcommand\email{\nolinkurl}

\title{\zhm\thanks{版本 1.0} 中文索引处理程序}
\author{刘海洋}
\date{2014 年 2 月 6 日}

\begin{document}

\maketitle

\section{命令行}

\begin{syntax}
\halign{#&#\hfil\cr
zhmakeindex &[-enc~<enc>] [-i] [-o~<output>] [-q] [-r] [-s~<style>]\cr
            &[-senc~<senc>] [-strict] [-t~<log>] [-z~<sort>]\cr
            &[<idx0> <idx1> <idx2> ...]\cr
}
\end{syntax}

\section{简介}

\zhm 是一个通用的中文多级索引处理程序,它从一个或多个输入文件读入索引项,将其
内容按指定的方式分组、排序,然后按格式将整理好的索引输出到文件。索引项可以有 3
个级别(0, 1, 2)的嵌套。\zhm 主要用于 \LaTeX{} 索引的处理,其功能和用法与
\pkg{makeindex} 相似,并支持中文的分组与排序。

输入与输出文件的格式由一个格式文件确定。默认的输入/输出是 ".idx"/".ind" 格式
的,即 \LaTeX{} 格式的索引文件。

如果没有显式指定,第一个输入文件(\meta{idx0})的主文件名将用于确定输出和日志
文件的主文件名。对每个输入文件名 \meta{idx0}, \meta{idx1}, \ldots, \zhm 会首先
查找这个名字的文件;如果找不到且文件名没有后缀,则加上 ".idx" 后缀查找此文件;
如果还找不到文件,\zhm 则会中止。

如果只有一个输入文件,并且没有显式用 "-s" 选项指定格式文件,\zhm 会使用后缀为
".mst" 的默认格式文件(如果存在的话)。


\section{选项说明}

\begin{description}
  \item[-enc~\meta{enc}] 设置输入输出文件的编码为 \meta{enc}。
  \item[-senc~\meta{senc}] 设置读入格式文件的编码为 \meta{senc}。
  \item[-strict] 严格区分不同 encapsulated 命令的页码。
  \item[-z~\meta{sort}] 设置中文排序分组方式为 \meta{sort}。
\end{description}

\section{格式文件}

\section{顺序}

\section{特殊效果}

\section{与 \pkg{makeindex} 的比较}

\section{版权与许可}
\index{版权}\index{许可}

版权所有:2014 年,刘海洋 \email{leoliu.pku@gmail.com}

\index{LPPL}
本作品可在《the \LaTeX{} Project Public License》1.3 或更高版本的条件下发布与
修改。最新版本的 LPPL 许可证可以在
\begin{quote}
  \url{http://www.latex-project.org/lppl.txt}
\end{quote}
下载;该许可证同时也包含在所有最新的 \LaTeX{} 发行版中。

本作品目前处于 LPPL 维护状态“author-maintained”。

当前维护者是刘海洋。

\index{源文件}
本作品包括 \zhm 的程序及文档,由如下源文件:
\begin{verbatim}
dist.cmd
examples
input.go
main.go
numberedreader.go
output.go
radicalstrokes.go
radical_collator.go
readings.go
reading_collator.go
sorter.go
strokes.go
stroke_collator.go
style.go
style_test.go
doc/zhmakeindex.tex
kpathsea/dynamic_other.go
kpathsea/dynamic_windows_386.go
kpathsea/kpathsea.go
maketables/make-table.cmd
maketables/maketables.go
\end{verbatim}
以及编译源文件得到的二进制文件 \path{zhmakeindex.exe}、PDF 文档
\path{zhmakeindex.pdf} 组成。

\index{Unicode}
大部分汉字数据来自 Unicode 项目(\url{http://www.unicode.org/}):
\begin{verbatim}
maketables/CJKRadicals.txt
maketables/Unihan_DictionaryLikeData.txt
maketables/Unihan_RadicalStrokeCounts.txt
maketables/Unihan_Readings.txt
\end{verbatim}

\index{海峰五笔}
部分字形数据来自海峰五笔项目(\url{http://okuc.net/sunwb/}):
\begin{verbatim}
maketables/sunwb_strokeorder.txt
\end{verbatim}

\bibliography{zhmakeindex}

\printindex

\end{document}

vim:tw=78
